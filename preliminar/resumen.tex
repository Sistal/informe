\chapter*{RESUMEN}
\addcontentsline{toc}{chapter}{RESUMEN}

\begin{comment}
    El resumen debe dar cuenta en forma clara y simple el contenido de la obra, permite determinar la pertinencia del trabajo y decidir al lector si el documento es de su interés. Se constituye de una descripción simple, breve y concisa de:

    - El objetivo del trabajo
    - Método o procedimiento utilizado
    - Conclusiones o resultados obtenidos

    El resumen debe ser informativo y expresar en el mínimo número de palabras la mayor cantidad de información posible sobre el contenido del trabajo de titulación y su extensión máxima es de una página.

    también incluye las palabras claves, que son términos temáticos que más destacan del trabajo, los cuales permiten su búsqueda e identificación en distintos recursos de información, como catálogos de biblioteca, motores de búsqueda y bases de datos.
\end{comment}

El presente proyecto aborda la reingeniería del ``Sistema de Tallajes'' (SISTAL), un sistema web de gestión de uniformes corporativos desarrollado por la empresa GyV Inversiones para apoyar la administración de la dotación de vestimenta del personal en diferentes organizaciones. El sistema opera actualmente bajo una arquitectura monolítica basada en tecnologías obsoletas y sin prácticas modernas de desarrollo y despliegue, lo que limita su escalabilidad, mantenibilidad y capacidad de evolución.

El objetivo del proyecto es rediseñar la arquitectura de SISTAL para soportar de forma segura, escalable y mantenible el ciclo de gestión de uniformes, preservando los flujos de negocio existentes. Para ello, se adopta un enfoque basado en microservicios, prácticas DevOps y tecnologías en la nube, aplicando una metodología de reingeniería de software que incluye las fases de ingeniería inversa, rediseño arquitectónico e ingeniería directa. 

Como resultado, se definió una arquitectura modernizada y se desarrolló un prototipo funcional del módulo de funcionario, el cual permitió validar la propuesta mediante pruebas funcionales y de usabilidad. La solución propuesta demuestra mejoras relevantes en modularidad, escalabilidad y mantenibilidad, confirmando la viabilidad técnica y el valor estratégico de la modernización planteada.

\noindent \textbf{PALABRAS CLAVE}

Ingeniería de software, Reingeniería de software, Modernización tecnológica, Arquitectura de software, Microservicios, Computación en la nube, DevOps, Sistema de tallajes, Industria textil, Contenedores, Kubernetes.



\newpage

\chapter*{ABSTRACT}

This project addresses the reengineering of ``Sistema de Tallajes'' (SISTAL), a web-based corporate uniform management system developed by GyV Inversiones to support organizations in administering employee uniform allocation. The system currently operates under a monolithic architecture based on obsolete technologies and lacks modern development and deployment practices, which limits its scalability, maintainability, and capacity for evolution.

The objective of the project is to redesign SISTAL’s architecture to securely, scalably, and maintainably support the uniform management lifecycle while preserving existing business workflows. To this end, an approach based on microservices, DevOps practices, and cloud technologies is adopted, applying a software reengineering methodology that includes the phases of reverse engineering, architectural redesign, and forward engineering. As a result, a modernized architecture was defined and a functional prototype of the employee module was developed, allowing the proposal to be validated through functional and usability testing. The proposed solution demonstrates notable improvements in modularity, scalability, and maintainability, confirming both the technical feasibility and the strategic value of the proposed modernization.

\noindent \textbf{KEYWORDS}

Software engineering, Software reengineering, Technological modernization, Software architecture, Microservices, Cloud computing, DevOps, Sizing system, Textile industry, Containers, Kubernetes.
