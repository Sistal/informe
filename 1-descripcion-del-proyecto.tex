\chapter{DESCRIPCIÓN GENERAL DEL PROYECTO}

Este capítulo presenta la descripción del proyecto y la identificación del módulo que será desarrollado como prototipo final, junto con sus objetivos, alcances y limitaciones y los recursos utilizados .

\section{Descripción del Proyecto}

% Describir proyecto e introducir a la sección siguiente

En línea con lo expuesto en la introducción, el proyecto se centra en la reingeniería de “Sistema de Tallajes” (SISTAL), comenzando con un diagnóstico exhaustivo del sistema monolítico actual, orientado a identificar sus principales limitaciones técnicas y estructurales. A partir de este análisis, se plantea el diseño de una arquitectura orientada a la nube basada en \textbf{microservicios}, enfoque que, a diferencia del modelo monolítico, divide el sistema en múltiples componentes independientes que se comunican entre sí, cada uno con una funcionalidad específica. 

Este enfoque se complementa con el uso de \textbf{contenedores}, entendidos como unidades que encapsulan una aplicación, garantizando entornos de ejecución consistentes y reutilizables, y con la \textbf{orquestación} de estos a través de Kubernetes, tecnología que administra automáticamente cada contenedor, creando y destruyéndolos basado en la carga del sistema. Finalmente, se incorpora la adopción de prácticas \textbf{\textit{DevOps}} incluyendo integración y despliegue continuo (\textit{CI/CD}), prácticas orientadas tanto a la automatización de los procesos de despliegue y actualización del sistema como a la optimización del proceso de desarrollo, mediante la estandarización de flujos de trabajo, la ejecución automática de pruebas para la detección temprana de errores.

La implementación se realiza de forma incremental, comenzando exclusivamente por el Módulo del Funcionario como alcance para este proyecto. SISTAL cuenta con tres módulos principales asociados a los actores clave del sistema (Administrador, Funcionario y Proveedor) tal como se muestra en la \autoref{fig:modulo-a-desarrollar}.

\begin{figure}[H]
    \centering
    \includegraphics[width=\textwidth]{figuras/diagramas-actuales/modulo-a-desarrollar}
    \caption{Interacción de los actores con los módulos del sistema, encerrado en un cuadro verde se encuentra el módulo a implementar.}
    \sourcefig{Diseño Propio}{}{}
    \label{fig:modulo-a-desarrollar}
\end{figure}

La descripción detallada de estos actores y de sus respectivos módulos se presenta en la \autoref{subsec:actores-principales} del \autoref{chap:descripcion-empresa-sistema}.

Se seleccionó el módulo de Funcionario como punto de partida porque constituye la base operativa del sistema: es el primero en interactuar con el flujo de dotación, registra la información crítica de tallas y habilita los procesos posteriores de aprobación, confección y despacho.


Se construirán los componentes/microservicios destinados a permitir a un funcionario registrar sus tallas y gestionar sus solicitudes. Al cierre, se realizarán pruebas funcionales y de usabilidad, validando criterios acordados y contrastando los resultados con la línea base del enfoque monolítico original.

\section{Objetivos}

\subsection{Objetivo General}

Diseñar la nueva arquitectura del sistema de gestión de uniformes corporativos SISTAL, basada en microservicios y tecnologías en la nube, que permita mejorar la escalabilidad y mantenibilidad de la plataforma, y desarrollar el prototipo del módulo de funcionario.

\subsection{Objetivos Específicos}

\begin{itemize}
    \item Analizar la arquitectura actual del sistema monolítico, identificando limitaciones y oportunidades de mejora en escalabilidad y mantenibilidad.
    \item Documentar los procesos de negocio y flujos de datos existentes con el propósito de modelarlos y replicarlos en la nueva arquitectura del sistema.
    \item Diseñar una nueva arquitectura basada en microservicios escalable en la nube.
    \item Proponer lineamientos de infraestructura, tecnologías y prácticas DevOps.
    \item Desarrollar el prototipo del módulo de funcionario del sistema, a modo de implementación inicial y validación de la propuesta.
\end{itemize}

\section{Alcances y Limitaciones}

\subsubsection{Alcances}

\begin{itemize}
    \item Reestructurar el sistema monolítico de SISTAL hacia una arquitectura de microservicios.
    \item Diseñar los microservicios centrales (usuarios, pedidos, inventario y estados).
    \item Documentar la configuración de infraestructura en nube (Google Cloud Platform).
    \item Implementar integración y despliegue continuo (CI/CD) y buenas prácticas DevOps.
    \item Desarrollar el prototipo del módulo de funcionario del sistema.
\end{itemize}

\subsubsection{Limitaciones}

\begin{itemize}
    \item No se contempla el desarrollo de aplicaciones móviles nativas o complementarias.
    \item No se usarán datos reales.
    \item No habrá capacitación masiva a usuarios finales.
    \item No se implementarán integraciones con sistemas externos no contemplados en la versión original.
    \item Se desarrollará únicamente el módulo de funcionario.
\end{itemize}

\section{Recursos}

\subsection{Recursos Humanos}

El proyecto cuenta con el siguiente equipo de trabajo presentado en la \autoref{tab:recursos-humanos}.

\begin{table}[H]
    \centering
    \small
    \setstretch{1}
    
    \caption{Recursos Humanos del Proyecto}
    \sourcefig{Elaboración Propia}{}{}
    \label{tab:recursos-humanos}
    \begin{tabularx}{\textwidth}{l X X}
        \toprule
        \textbf{Rol} & \textbf{Responsabilidades principales} & \textbf{Nombre(s)} \\
        \midrule
        Product Owner &
        Definición de la visión del producto, priorización y gestión de requerimientos, y alineación con las necesidades del negocio. &
        Sr.~Julio Gómez \\
        \addlinespace
        Equipo de Desarrollo &
        Diseño, desarrollo, implementación y mantenimiento de la solución. &
        \parbox[t]{0.31\textwidth}{ \raggedright
            Gerardo Araneda Mella,
            Andrés Gómez Rodríguez,
            Nicolás Jiménez Romero.
        } \\
        \bottomrule
    \end{tabularx}
\end{table}

\subsection{Recursos de Hardware}

\begin{itemize}
    \item Computadora con capacidad para ejecutar entornos de desarrollo, contenedores y pruebas locales.
    \item Servidores en la nube proporcionados por el proveedor seleccionado.
\end{itemize}

\subsection{Recursos de Software}

Los recursos presentados en la \autoref{tab:recursos-de-software} representas las herramientas y tecnologías utilizadas para la realización del proyecto, una justificación más detallada se detalla en capítulos posteriores.

\begin{table}[H]
    \centering
    \small
    \setstretch{1}
    
    \caption{Recursos de Software del Proyecto}
    \sourcefig{Elaboración Propia}{}{}
    \label{tab:recursos-de-software}
    \begin{tabularx}{\textwidth}{X X}
        \toprule
        \textbf{Componente} & \textbf{Herramienta / Tecnología Utilizada} \\
        \midrule
        Lenguaje Frontend &
        React \\
        \addlinespace
        Lenguaje Backend &
        Go \\
        \addlinespace
        Contenedorización y Orquestación &
        Docker y Kubernetes \\
        \addlinespace
        Plataforma Cloud &
        GCP \\
        \addlinespace
        Base de datos &
        PostgreSQL \\
        \addlinespace
        Control de versiones &
        Git/GitHub \\
        \addlinespace
        CI/CD &
        GitHub Actions \\
        \bottomrule
    \end{tabularx}
\end{table}
