\chapter*{Introducción}
\addcontentsline{toc}{chapter}{INTRODUCCIÓN}

La reingeniería de software corresponde al proceso de analizar, reestructurar y modernizar un sistema existente con el propósito de mejorar su diseño, su rendimiento y su mantenibilidad, preservando las funcionalidades esenciales que sustentan el negocio.
En este contexto, el presente trabajo aborda la modernización del ``Sistema de Tallajes'' (SISTAL), un sistema \textit{web} desarrollado por la empresa GyV Inversiones en el año 2009, para gestionar la dotación de uniformes corporativos en organizaciones que requieren administrar de forma centralizada las tallas, solicitudes, confección, despacho y trazabilidad de las prendas entregadas a su personal.

Al hablar de diseño arquitectónico y tecnológico, se hace referencia a la planificación y estructuración de los componentes de \textit{software} y las tecnologías que los soportan, estableciendo la forma en que estos se organizan, se comunican e interactúan entre sí para dar cumplimiento a los objetivos del sistema. Por su parte, una aplicación, plataforma o sistema \textit{web} se define como una solución informática accesible a través de un navegador \textit{web}, que permite a los usuarios interactuar con servicios y datos mediante internet o una red interna.
En consecuencia, en este documento se hará referencia a SISTAL como un “Sistema \textit{Web}”.

A lo largo de más de una década de operación, SISTAL ha desempeñado un rol fundamental como herramienta de apoyo para grandes empresas. Sin embargo, su arquitectura monolítica (un diseño en el que todos los componentes del sistema se encuentran estrechamente acoplados), basada en las tecnologías PHP 5.6 y MySQL, y alojada en un servicio de \textit{hosting} compartido, presenta actualmente limitaciones significativas en términos de escalabilidad, mantenibilidad, seguridad y capacidad de evolución. A ello se suma la ausencia de estándares de diseño, la escasa adopción de buenas prácticas de desarrollo y la inexistencia de procesos de despliegue controlados, factores que dificultan la incorporación de nuevas funcionalidades y aumentan los riesgos operativos del sistema.

Por lo tanto, el proyecto se enfoca en el rediseño arquitectónico y tecnológico de este sistema \textit{web}, definiendo una base tecnológica sólida para futuras innovaciones y expansiones de la plataforma, soportando el ciclo completo de dotación de uniformes y preservando los flujos de negocio existentes. A modo de validación de esta propuesta, se realiza el desarrollo del módulo de funcionarios del sistema, componente clave del flujo de dotación, por ser el responsable de registrar tallas, gestionar solicitudes y habilitar los procesos subsiguientes de aprobación, confección y despacho.


El desarrollo de este proyecto surge de la motivación por profundizar en el campo de la arquitectura de software, junto con nuestra cercanía al dominio y una comprensión detallada de las reglas de negocio vinculadas a la gestión de uniformes corporativos. Esta perspectiva, construida a partir del contacto directo con quienes han participado en la operación y el diseño del sistema en el pasado, nos permite identificar con precisión los puntos críticos, priorizar mejoras con un enfoque práctico y proponer soluciones viables que respondan a las necesidades reales de la operación.

Además, nos motiva la oportunidad de generar un impacto significativo en un sector donde escasean soluciones integrales comparables. Modernizar y fortalecer SISTAL constituye un desafío formativo de gran valor, ya que nos enfrenta a problemáticas de alta complejidad y, al mismo tiempo, nos brinda la posibilidad de aplicar los conocimientos teóricos adquiridos, que hasta ahora no habíamos podido poner en práctica durante nuestra etapa formativa. Asimismo, el proyecto nos permite aportar resultados concretos a un sistema de uso empresarial, mejorando su trazabilidad, eficiencia operativa y capacidad de evolución tecnológica. En síntesis, elegimos este proyecto porque integra el conocimiento del negocio con una oportunidad real de aprendizaje y de impacto positivo en la organización.

% Metodología

La metodología aplicada en este proyecto se basa en un enfoque inspirado de la reingeniería de software, el cual comprende tres fases principales: Ingeniería Inversa (\textit{Reverse Engineering}), Alteración (\textit{Alteration}) e Ingeniería Directa (\textit{Forward Engineering}). En la primera fase se analizó el sistema original para recuperar conocimiento y redocumentar su funcionamiento, arquitectura, procesos y dependencias internas. Posteriormente, en la etapa de alteración, se llevó a cabo la reorganización y modernización conceptual del sistema mediante actividades de reestructuración y rediseño arquitectónico basadas en principios de microservicios, computación en la nube y prácticas DevOps. Finalmente, la fase de ingeniería directa correspondió al desarrollo del Módulo del Funcionario dentro de la arquitectura propuesta, utilizando el marco de trabajo Scrum para gestionar el proceso de implementación mediante iteraciones cortas y validaciones continuas.

% Presentar un resumen del contenido de los diferentes capítulos.

Los primeros capítulos del informe presentan el contexto general del proyecto, incluyendo la introducción y descripción del mismo, la historia de la empresa, así como la descripción y el análisis del sistema. En esta última sección se revisan aspectos como su arquitectura, la estructura de la base de datos, los flujos de información, entre otros componentes relevantes. Todo ello permite establecer los antecedentes y comprender la problemática que motiva la necesidad de una reingeniería. Posteriormente, se desarrolla el marco teórico que sustenta el proyecto, incorporando conceptos esenciales y otros enfoques relevantes que contribuyen a la propuesta técnica. Con esta base, se pasa al desarrollo del proyecto, donde se especifican los requerimientos, el diseño propuesto para el nuevo sistema, las herramientas y metodologías seleccionadas, finalizando con la documentación final, la implementación y conclusiones.
