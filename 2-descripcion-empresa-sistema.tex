\chapter{DESCRIPCIÓN DE LA EMPRESA Y EL SISTEMA} \label{chap:descripcion-empresa-sistema}

Este capítulo presenta la descripción de la empresa responsable del desarrollo del sistema, revisando su funcionamiento y modelo de negocio, con el propósito de contextualizar el origen y los objetivos del sistema dentro del ámbito organizacional. Posteriormente, se describe de manera general el sistema actual, identificando sus principales actores, flujos de información y procesos operativos, lo que permite comprender el rol que cumple SISTAL en la gestión de uniformes corporativos y las dinámicas de interacción entre los distintos usuarios que lo utilizan.

\section{Descripción de la Empresa}

\subsection{Historia}

En el año 2005, el ingeniero civil industrial Julio Gómez Vega, se integra a la empresa de la industria textil ``La Scala'', dedicada a la producción y comercialización de prendas de vestir, asumiendo el rol de ingeniero de planificación de producción con el objetivo principal de desarrollar el departamento de venta y distribución de uniformes corporativos para sus diferentes clientes. Tras unos años, enfocado en automatizar y centralizar flujos de datos, dirige el desarrollo de una plataforma informática que permite obtener de forma rápida y eficientemente la información necesaria para producción y logística de cada prenda correspondiente a los uniformes comprometidos para la distribución.  La herramienta que impulsó permitía gestionar curvas de talla, coordinar la logística de entrega y recopilar retroalimentación en la postventa de los productos suministrados.

Tras el cierre de La Scala, debido a diferencias entre sus propietarios, Julio Gómez y Sebastián Gómez fundaron GyV Inversiones Ltda. en el año 2009, con el propósito de ofrecer soluciones tecnológicas a la industria textil. Aprovechando la experiencia adquirida, desarrollaron el Sistema de Tallajes (SISTAL), una plataforma que automatizó la toma de medidas, la generación de pedidos y la distribución de uniformes. Con el tiempo, SISTAL se consolidó como una herramienta clave en grandes empresas como BancoEstado y Banco de Chile.

Durante más de una década, SISTAL funcionó exitosamente como sistema de administración de uniformes. Sin embargo, los avances tecnológicos y las crecientes exigencias en materia de seguridad de la información hicieron necesaria una asesoría externa, con el fin de actualizar la plataforma e incorporar nuevos conceptos tecnológicos. A continuación, se presentan la Misión y Visión de la empresa.

\subsection{Misión}

\begin{quote}
    Ser una empresa líder dentro del rubro textil, en lo que a sistemas de información se refiere, privilegiando la veracidad, rapidez y automatización de la información, con sistemas de bajo costo y fácil uso. Apostamos a la diferenciación en la comercialización mediante un servicio orientado al cliente final, con atención personalizada a cada usuario. Seremos considerados por nuestros clientes, más que un proveedor de servicios, un aliado estratégico que les ayudará a concentrarse específicamente en su negocio, dejando la administración de los uniformes corporativos en nuestras manos.
\end{quote}

\subsection{Visión}

\begin{quote}
    Formar parte de las mejores empresas de gestión de uniformes corporativos a nivel nacional e internacional, con modelos de vanguardia, una confección que cumpla con todos los estándares de calidad del mercado, y un servicio integral que exceda las expectativas de nuestros clientes. Para lograrlo, nuestro foco está apuntando a dos principios básicos, en primer lugar, potenciar las fortalezas de nuestro recurso humano, considerándolo como nuestro principal activo, y por supuesto, comprometernos en ir adoptando continuamente mejoras tecnológicas a nuestros sistemas informáticos.
\end{quote}

\subsection{Modelo de Negocios}

GyV Inversiones Ltda. se dedica al desarrollo, comercialización y arriendo de sistemas de gestión orientados principalmente a la industria textil. Desde su fundación en 2009, la empresa ha centrado sus esfuerzos en ofrecer soluciones informáticas que optimizan los procesos productivos y administrativos de las compañías del rubro, evolucionando hacia plataformas que integran a la gran empresa, sus colaboradores y proveedores. Su principal línea de negocio se enfoca en proveer sistemas especializados en la gestión de uniformes corporativos, destacando la plataforma SISTAL (Sistema de Tallas y Logística), diseñada para facilitar la administración y coordinación entre las empresas que comercializan uniformes, sus clientes y usuarios finales.

Los ingresos obtenidos por GyV Inversiones Ltda., corresponde principalmente al arriendo de plataformas de gestión, soporte informático de sistemas a medida, asesorías en diversos sistemas ERP como Softlan, Random, Manager y otras asesorías en Access y Excel.

El modelo de negocios actual presenta algunas características primordiales para el desarrollo de la operación. Una de las más importantes es el equipo de trabajo, demostrando alto profesionalismo en términos técnicos y experiencia en las actividades del negocio, la propuesta de valor tiene por objetivo ofrecer un servicio integral y de excelencia a los clientes, comprendiendo sus necesidades para ofrecerles la solución más eficiente para sus problemáticas. Se puede destacar además la mejora continua con respecto a los costos de la operación, privilegiando en lo posible el reacondicionamiento de equipos y reciclaje de insumos, permitiendo tener una estructura mínima de costos. 

Basándonos en las características identificadas, se diseñó un esquema del modelo de negocios que se muestra en la \autoref{fig:lienzo-canva-actual}, que permite tener una vista completa del modelo de negocios e identificar el valor que entrega el proyecto a la organización.

\begin{figure}[H]
    \centering
    \includegraphics[width=\textwidth]{figuras/diagramas-actuales/lienzo-canva}
    \caption{Lienzo Canva del modelo de negocios actual de la empresa.}
    \sourcefig{Diseño Propio}{}{}
    \label{fig:lienzo-canva-actual}
\end{figure}

Como se observa, la propuesta de valor se conecta con todo el modelo de negocios: los clientes buscan precisamente una solución que centralice la gestión, reduzca retrasos y mejore la comunicación, por lo que el sistema se vuelve atractivo para áreas de Recursos Humanos y proveedores de confección. Asimismo, el funcionamiento del sistema depende de la participación coordinada de instituciones que requieren dotación uniformada, proveedores de confección y servicios de hosting. Estos socios aportan insumos, logística y soporte técnico que permiten que la automatización prometida sea efectiva y genere valor para todos los involucrados.

Para cumplir esa promesa, el conocimiento del rubro, infraestructura tecnológica y la experiencia acumulada, permite entregar un servicio confiable a través de una plataforma web accesible. Esto también determina los costos del proyecto, centrados en la tecnología, y la forma en que se generan ingresos, mediante arrendamiento, implementación y soporte.


\subsection{Análisis FODA}

En la \autoref{tab:analisis-foda}, se muestra el análisis FODA realizado, que lista los atributos identificados de la empresa que afectan el desarrollo y la sostenibilidad del sistema, como del modelo de negocio asociado.

\begin{table}[H] % la sintaxis de tablas de latex es media basura xd
    \footnotesize
    \centering % centra la tabla
    \singlespacing % interlineado 1 dentro de la tabla
    \setlength\arrayrulewidth{1pt}
    \renewcommand{\arraystretch}{1.5} % espacio de 1.5 entre filas
    \setlist[itemize]{leftmargin=*, itemsep=0pt, topsep=0pt}
    \newcolumntype{Y}{>{\centering\arraybackslash}X} % no pescar mucho jasjjas

    \caption{Análisis FODA con principales atributos identificados de la empresa}
    \sourcefig{Elaboración Propia}{}{}
    \label{tab:analisis-foda}
    
    \begin{tabularx}{\textwidth}{|Y|Y|}
        \hline
        \rowcolor{celeste-claro} \textbf{Fortalezas} & \textbf{Oportunidades} \\
        \hline

        \begin{itemize}
            \item Plataforma integral que centraliza datos y procesos.
            \item Propuesta de valor clara: eficiencia operativa y toma de decisiones basada en información.
            \item Integración fluida con socios clave (proveedores tecnológicos, instituciones externas, sistemas colaboradores).
            \item Conocimiento profundo del dominio del negocio.
        \end{itemize} &
        \begin{itemize}
            \item Crecimiento de la demanda por sistemas integrados y analíticos.
            \item Posibilidad de expandir el ecosistema con nuevos socios y alianzas o soporte de nuevos productos.
            \item Tendencia del mercado a la digitalización de procesos.
            \item Interés de clientes en reducir costos operativos mediante plataformas centralizadas.
        \end{itemize}
        \\ \hline

        \rowcolor{celeste-claro} \textbf{Debilidades} & \textbf{Amenazas} \\
        \hline
        \begin{itemize}
            \item Complejidad técnica que dificulta la incorporación de nuevos usuarios o mantenedores.
            \item Procesos internos que aún requieren formalización o documentación.
            \item Deuda técnica y limitación de escalabilidad.
        \end{itemize} & 
        \begin{itemize}
            \item Cambios constantes en estándares tecnológicos o regulatorios.
            \item Riesgo de dependencia excesiva de algunos socios clave.
            \item Expectativas crecientes de los clientes respecto a tiempos y calidad del servicio.
            \item Vulnerabilidades inherentes a cualquier sistema digital frente a ciberataques.
        \end{itemize}
        \\ \hline
    \end{tabularx}
\end{table}

A partir del análisis previamente expuesto, se construyó una matriz FODA presentada en la \autoref{tab:matriz-foda}, que cruza las variables internas y externas, para derivar un conjunto de estrategias orientadas a fortalecer la posición de la organización. La matriz resultante constituye una guía estratégica para la toma de decisiones y la evolución futura del sistema.

\begin{table}[H]
    \footnotesize
    \singlespacing
    \setlength\arrayrulewidth{1pt}
    \renewcommand{\arraystretch}{1.5}
    \setlist{leftmargin=*, itemsep=0pt}

    \caption{Matriz FODA con análisis estratégicos}
    \sourcefig{Elaboración Propia}{}{}
    \label{tab:matriz-foda}

    \begin{tabularx}{\textwidth}{|>{\columncolor{celeste-claro}}p{1em}|X|X|}
        \hline
        \rowcolor{celeste-claro} & \textbf{Fortalezas} & \textbf{Debilidades} \\
        \hline

        \rotatebox[origin=r]{90}{\textbf{Oportunidades}} & 
        \textbf{Estrategias FO:}
        \begin{itemize}
            \item  Usar la integración fluida y el conocimiento del negocio para acelerar la creación de módulos o conectores que permitan sumar nuevos aliados, diversificar servicios y ampliar la cobertura del ecosistema.
        \end{itemize} & 
        \textbf{Estrategias DO:}
        \begin{itemize}
            \item La tendencia hacia sistemas integrados y fáciles de usar justifica inversiones en interfaces más intuitivas, automatización interna y reducción de complejidad técnica.
            \item La incorporación de nuevos actores exige estándares más altos de orden y documentación, lo cual ayuda a superar brechas internas existentes.
        \end{itemize} \\
        \hline

        \rotatebox[origin=r]{90}{\textbf{Amenazas}} & 
        \textbf{Estrategias FA:}
        \begin{itemize}
            \item La plataforma puede responder con mayor agilidad a cambios normativos al tener procesos y datos unificados, reduciendo impacto y tiempos de adaptación.
        \end{itemize}
        & 
        \textbf{Estrategias DA:}
        \begin{itemize}
            \item Fortalecer la arquitectura para evitar riesgos frente a regulaciones nuevas, aumentos de demanda o exigencias de rendimiento.
            \item Disminuir complejidad interna y mejorar orden técnico para reducir la superficie de ataque y fortalece la resiliencia del sistema.
        \end{itemize}
        \\ \hline
    \end{tabularx}
\end{table}

El análisis FODA y la matriz estratégica permiten obtener una visión clara del estado actual de la organización y del sistema, como de las rutas de acción más adecuadas para potenciar su desarrollo.

\section{Descripción del Sistema (SISTAL)}

Como se ha mencionado anteriormente, SISTAL es un sistema web de gestión de uniformes corporativos, desarrollado para atender a organizaciones que requieren administrar de manera eficiente la dotación de uniformes a su personal. Este sistema cubre todo el flujo natural de dotación de uniformes, desde el registro de tallas de los funcionarios, hasta la confección y despacho del proveedor de uniformes. Para comprender bien todo este flujo, se comenzará explicando sobre los principales actores que interactúan en este sistema y sus respectivos módulos.


\subsection{Actores Principales y sus Módulos} \label{subsec:actores-principales}

El sistema considera \textbf{tres actores principales}, denominados \textbf{Funcionario}, \textbf{Administrador} y \textbf{Proveedor}, los cuales interactúan tanto al interior de las organizaciones como dentro de la propia plataforma. Cada uno de ellos cumple un rol específico y dispone de una interfaz o módulos particulares que facilitan sus tareas dentro del proceso de gestión de uniformes corporativos. En la \autoref{fig:actores-del-sistema} se presenta un diagrama que representa cómo estos actores interactúan con el sistema.

\begin{figure}[H]
    \centering
    \includegraphics[width=\textwidth]{figuras/actores-del-sistema}
    \caption{Diagrama explicativo de actores del sistema}
    \sourcefig{Diseño Propio}{}{}
    \label{fig:actores-del-sistema}
\end{figure}

Como se señaló previamente, los actores interactúan con una \textbf{Interfaz Gráfica de Usuario, o Módulo Funcional}, la cual se encuentra directamente asociada al rol que desempeña cada actor y que agrupa las tareas que cada uno puede realizar. A continuación se explican los tres actores y las funciones que puede realizar en su respectivo módulo.

\begin{itemize}
    \item \textbf{Funcionario:} Representa al trabajador o personal de la organización que \textbf{utiliza uniformes corporativos como parte de su vestimenta laboral}. A través del módulo Funcionario de SISTAL, este usuario puede registrar y actualizar sus medidas de talla desde cualquier dispositivo con conexión a internet\footnote{Al ser un sistema web, se puede acceder desde cualquier dispositivo que soporte un navegador, siendo una de las propuestas de valor iniciales del sistema.}, permitiéndole revisar la información de sus prendas asignadas y solicitar cambios o reposiciones cuando sea necesario.

    \item \textbf{Proveedor}: Representa a un trabajador de la empresa externa encargada de la \textbf{confección y entrega de los uniformes corporativos}. A travéz del módulo Proveedor de SISTAL, el proveedor puede acceder a la información consolidada de tallas, cantidades y tipos de prendas requeridas, la cual puede descargar o consultar en línea para gestionar la producción. También puede actualizar el estado de los pedidos, registrar despachos y facturas.

    \item \textbf{Administrador}: Representa al trabajador de la institución que requiere dotar de uniformes a su personal y que cuenta con privilegios de administración a travéz del módulo Administrador de SISTAL. Es el responsable de \textbf{solicitar uniformes al proveedor, así como de gestionar y supervisar todo el proceso de dotación dentro de la organización solicitante}. Sus funciones incluyen gestionar funcionarios, controlar pedidos, aprobar solicitudes, generar reportes y mantener actualizada la información institucional. Asimismo, coordina la comunicación con el proveedor para asegurar la trazabilidad de los procesos y el cumplimiento de las políticas de dotación.
\end{itemize}

%Cada actor puede acceder a su respectivo módulo desde una pantalla de inicio de sesión (\autoref{fig:interfaz-login}), donde el sistema reconoce el rol del actor y le muestra el módulo correspondiente. Normalmente, un funcionario ingresaría con su RUT en el campo de Usuario, pero en caso de ser un Proveedor o Administrador, este tiene un Usuario específico.

La \autoref{fig:interfaz-login} presenta la pantalla de inicio actual del sistema, cada actor puede acceder a su respectivo módulo desde aquí. Cuando uno de ellos ingresa sus credenciales (Usuario y Contraseña), el sistema automáticamente reconoce el su rol y le muestra el módulo correspondiente. Actualmente, un funcionario ingresaría con su RUT como su credencial de usuario, en caso de ser un Administrador o Proveedor, este tiene sus credenciales de usuario específicas.

\begin{figure}[H]
    \centering
    \includegraphics[width=\textwidth, frame=1pt]{figuras/interfaz-login.png}
    \caption{Página de inicio de sesión. Presenta los campos de Usuario, Contraseña y Recuperar contraseña con un \textit{email} registrado.}
    \sourcefig{Interfaz de SISTAL alojada en un entorno de pruebas}{}{}
    \label{fig:interfaz-login}
\end{figure}

Las figuras \autoref{fig:interfaz-funcionario}, \autoref{fig:interfaz-proveedor} y \autoref{fig:interfaz-administrador} presentan una vista de las interfaces gráficas de cada módulo con motivos ilustrativos, todos los módulos tienen un menú a la izquierda con botones para acceder a las distintas funciones, algunos de estos botones despliegan nuevos menús al pasar el cursor por encima. una explicación más a fondo de estos módulos y sus funciones se encuentra en el \autoref{chap:analisis-tecnico-del-sistema-actual}, \autoref{sec:diagrama-casos-de-uso-actual}.

\begin{figure}[p]
    \centering
    \includegraphics[width=\textwidth, frame=1pt]{figuras/interfaz-funcionario.png}
    \caption{Interfaz del módulo de funcionario. Muestra datos del funcionario, su sucursal de la empresa y sus tallas que puede editar.}
    \sourcefig{Interfaz de SISTAL alojada en un entorno de pruebas}{}{}
    \label{fig:interfaz-funcionario}
\end{figure}

\begin{figure}[p]
    \centering
    \includegraphics[width=\textwidth, frame=1pt]{figuras/interfaz-proveedor.png}
    \caption{Interfaz del módulo de proveedor. Muestra los funcionarios enviados por el administrador de la empresa solicitante que les corresponde uniforme.}
    \sourcefig{Interfaz de SISTAL alojada en un entorno de pruebas}{}{}
    \label{fig:interfaz-proveedor}
\end{figure}

\begin{figure}[p]
    \centering
    \includegraphics[width=\textwidth, frame=1pt]{figuras/interfaz-administrador.png}
    \caption{Interfaz del módulo de administrador. El menú que se muestra permite al administrador aprobar o rechazar los funcionarios nuevos que se registraron en el sistema.}
    \sourcefig{Interfaz de SISTAL alojada en un entorno de pruebas}{}{}
    \label{fig:interfaz-administrador}
\end{figure}

\clearpage
\subsection{Procesos que Aborda SISTAL}

Para comprender el flujo de información dentro del sistema, en la \autoref{fig:flujo-simple-sistema} se presenta, de forma general y simplificada, el proceso de dotación de uniformes al personal de una empresa cliente. Una explicación más detallada de estos flujos, junto con sus diagramas específicos, se encuentra en el \autoref{chap:analisis-tecnico-del-sistema-actual}, \autoref{sec:diagrama-de-flujo-actual}.

\begin{figure}[H]
    \centering
    \includegraphics[width=\textwidth]{figuras/flujo-simple-sistema}
    \caption{Flujo simplificado del proceso de dotación de uniformes apoyado por SISTAL}
    \sourcefig{Diseño Propio}{}{}
    \label{fig:flujo-simple-sistema}
\end{figure}

El proceso se inicia cuando un funcionario requiere un uniforme y se registra en SISTAL para ingresar sus tallas correspondientes. Una vez enviadas, estas quedan almacenadas en el sistema junto con las del resto de los funcionarios, lo que permite al administrador continuar con la \textbf{solicitud de confección}. En esta etapa, se remite al proveedor toda la información consolidada sobre modelos, cantidades y medidas necesarias para la elaboración de las prendas.

Luego de la confección de los uniformes, el proveedor inicia el \textbf{despacho}, en este caso, los uniformes pueden ser enviados a la sucursal correspondiente del funcionario, pero puede darse el caso de que el funcionario pueda retirar su uniforme en la misma instalación del proveedor, en este caso, el proveedor aprueba la recepción directamente.

Posterior a la entrega del uniforme, el funcionario puede solicitar un \textbf{arreglo}, representado en la \autoref{fig:flujo-simple-arreglo}.

\begin{figure}[H]
    \centering
    \includegraphics[width=\textwidth]{figuras/flujo-simple-arreglo}
    \caption{Flujo simplificado de solicitud de arreglo}
    \sourcefig{Diseño Propio}{}{}
    \label{fig:flujo-simple-arreglo}
\end{figure}

Este puede corresponder a dos tipos de procesos: el \textbf{cambio de prenda}, solicitado cuando alguna pieza del uniforme presenta un problema específico; y la \textbf{compostura}, mediante la cual se aplican ajustes o modificaciones a la misma prenda recibida. En este proceso es el funcionario quien directamente se comunica con el proveedor, dejando evidencia de las peticiones al administrador.

Con este repaso del sistema, del negocio al que entrega su solución y de su flujo de información, se dispone del contexto necesario para avanzar hacia el análisis técnico, instancia en la que se abordarán y examinarán en detalle las limitaciones del sistema.
