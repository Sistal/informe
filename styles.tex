\definecolor{light-gray}{gray}{0.95}
\definecolor{celeste}{HTML}{a5d8ff}
\definecolor{celeste-claro}{HTML}{e7f5ff}

% Parrafos con sangría de 5 espacios
\setlength\parindent{1.25cm}

\setlist{noitemsep,parsep=0pt,leftmargin=*}

\titlespacing*{\chapter}{0pt}{-30pt}{20pt}  % centrado, sin espacio arriba
\titlespacing*{name=\chapter,numberless}{0pt}{-30pt}{20pt}

\titleformat{\chapter}[display] % Estilo de los capítulos
  {\bfseries\large\centering}
  {\MakeUppercase\chaptername\ \thechapter}
  {5pt}
  {\titlerule[1pt]\vspace{-5pt}\MakeUppercase}

\titleformat{name=\chapter,numberless}[block]
  {\bfseries\large\centering}        % mismo estilo
  {}                                % sin número
  {0pt}                             % separación antes de la línea
  {\MakeUppercase}                  % título en mayúsculas

\titleformat{\section}[block]
  {\large\bfseries}
  {\thesection}
  {1em}
  {\MakeUppercase}

\titleformat{\subsection}[block]
  {\large\bfseries}
  {\thesubsection}
  {1em}
  {}

\titleformat{\subsubsection}[block]
  {\normalsize\bfseries}
  {\thesubsubsection}
  {1em}
  {}

% 'Capítulo X: ' en Índice
\renewcommand{\cftchappresnum}{CAPÍTULO~}
\renewcommand{\cftchapaftersnum}{:\ }
\renewcommand{\cftchapnumwidth}{7em}

\renewcommand\cftchapafterpnum{\setstretch{1}}
\renewcommand\cftsecafterpnum{\setstretch{1}}
\renewcommand\cftsubsecafterpnum{\setstretch{1}}
\renewcommand\cftfigafterpnum{\setstretch{1}}
\renewcommand\cfttabafterpnum{\setstretch{1}}

% Colores de Enlaces (Links)
\hypersetup{
    colorlinks,
    citecolor=black,
    filecolor=black,
    linkcolor=cyan,
    urlcolor=blue
}

% Configuración del estilo de código
\lstset{
    otherkeywords={<?php, ?>},
    basicstyle=\ttfamily\footnotesize,
    keywordstyle=\color{magenta}\bfseries,
    commentstyle=\color{gray}\itshape,
    stringstyle=\color{Green}\bfseries,
    numbers=left,
    numberstyle=\scriptsize,
    stepnumber=1,
    numbersep=8pt,
    breaklines=true,
    tabsize=4,
    showstringspaces=false
    aboveskip=5pt,
    belowskip=5pt,
    lineskip=-1pt,
    frame=tb
}

% Modificar Headers y Footers
\pagestyle{fancy}

%\setlength{\headheight}{35pt}

\fancyhf{}
\renewcommand{\headrulewidth}{0pt} % quitar línea superior
%\fancyhead[L]{\includegraphics[width=5cm]{figuras/marca-utem-horizontal.png}}
\fancyhead[L]{%
\begin{tikzpicture}[remember picture,overlay]
    \node[anchor=west, yshift=0.7cm, xshift=-4pt] at (0,0) {%
        \includegraphics[height=1.5cm]{figuras/logo-new-h}%
    };
\end{tikzpicture}
}
\fancyfoot[R]{\thepage}
\fancypagestyle{plain}{}

%%% COMANDOS PERSONALIZADOS %%% 

%%%%% 
\newcommand{\myrotcell}[1]{\rotcell{\makebox[0pt][l]{#1}}}

%%%%% \sourcefig{nombre}{año}{url} : para insertar una fuente a una figura
\newcommand{\sourcefig}[3]{%
    \vspace{-0.7em}
    \caption*{%
        \footnotesize\textsl{%
            Fuente: #1%
            \if\relax\detokenize{#2}\relax\else, #2\fi%
            \if\relax\detokenize{#3}\relax%
            % No hay URL → no mostrar "Enlace:"
            \else%
                , Enlace: \url{#3}%
            \fi%
        }%
    }%
}

%%%%% \code{} : para insertar una palabra de texto monoespaciada, como documentación
\newcommand{\code}[1]{\colorbox{light-gray}{\texttt{#1}}}
