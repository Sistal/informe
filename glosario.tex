\newglossary[glg]{negocio}{gls}{glo}{Glosario del Negocio}
\newglossary[glg]{tech}{tls}{tlo}{Glosario Técnico}

\makeglossaries

\newglossaryentry{corte}{
    type=negocio,
    name={Corte},
    description={Proceso interno de confección de uniformes de la entidad que provee los uniformes corporativos (proveedor)}
}
\newglossaryentry{segmento}{
    type=negocio,
    name={Segmento},
    description={Conjunto de prendas perteneciente a un área de desempeño en la empresa, por ejemplo vigilante, secretaria, etc. Masculinos y femeninos forman parte de segmentos distintos. Generalmente un proveedor se encarga de la confección de un solo segmento}
}
\newglossaryentry{envioacorte}{
    type=negocio,
    name={Envío a Corte},
    description={Solicitud que realiza el administrador de la institución que necesita uniformes al proveedor para indicar cuántos uniformes deben confeccionarse, basándose en la cantidad de funcionarios registrados con sus tallas ingresadas durante la temporada. En una misma temporada pueden generarse varias solicitudes. Dependiendo de la cantidad de funcionarios con tallas ingresadas, el administrador puede decidir cuándo enviar a corte y a cuántos funcionarios enviar}
}
\newglossaryentry{temporada}{
    type=negocio,
    name={Temporada},
    description={Período de tiempo definido por el administrador para organizar la solicitud y entrega de uniformes, por ejemplo una temporada de invierno y otra de verano, cada una con sus propios plazos y requerimientos}
}
\newglossaryentry{curvadetalla}{
    type=negocio,
    name={Curva de talla},
    description={Distribución de tallas por cada prenda}
}
\newglossaryentry{guiadedespacho}{
    type=negocio,
    name={Guía de despacho},
    description={Documento tributario electrónico destinado a acreditar el traslado de los uniformes desde la fábrica del proveedor a la respectiva sucursal definida}
}
\newglossaryentry{autenticacion}{
    type=tech,
    name={Autenticación},
    description={Proceso mediante el cual un sistema verifica la identidad de un usuario antes de permitir su acceso}
}
\newglossaryentry{autorizacion}{
    type=tech,
    name={Autorización},
    description={Proceso que determina los permisos y privilegios que tiene un usuario dentro del sistema}
}
\newglossaryentry{backend}{
    type=tech,
    name={Back-end},
    description={Conjunto de servicios, lógica de negocio y operaciones del sistema ejecutadas en el servidor}
}

\newglossaryentry{basededatosrelacional}{
    type=tech,
    name={Base de datos relacional},
    description={Modelo de almacenamiento que organiza información en tablas relacionadas mediante claves}
}

\newglossaryentry{bigballofmud}{
    type=tech,
    name={Big Ball of Mud},
    description={Antipatrón que describe sistemas sin arquitectura clara, con alto acoplamiento y baja mantenibilidad}
}

\newglossaryentry{cloudcomputing}{
    type=tech,
    name={Cloud Computing},
    description={Modelo de provisión de servicios computacionales a través de Internet bajo demanda}
}

\newglossaryentry{contenedor}{
    type=tech,
    name={Contenedor},
    description={Unidad portátil que empaqueta una aplicación junto con sus dependencias para ejecutarse de forma consistente en distintos entornos}
}

\newglossaryentry{desacoplamiento}{
    type=tech,
    name={Desacoplamiento},
    description={Principio de diseño que reduce dependencias entre componentes para facilitar su mantenimiento y evolución}
}
\newglossaryentry{despliegue}{
    type=tech,
    name={Despliegue},
    description={Proceso de publicar una aplicación o servicio en un entorno de ejecución}
}
\newglossaryentry{docker}{
    type=tech,
    name={Docker},
    description={Plataforma utilizada para crear y ejecutar contenedores de software}
}
\newglossaryentry{escalabilidad}{
    type=tech,
    name={Escalabilidad},
    description={Capacidad del sistema para aumentar su rendimiento o recursos según la demanda}
}
\newglossaryentry{frontend}{
    type=tech,
    name={Front-end},
    description={Interfaz gráfica y capa de presentación con la que interactúa directamente el usuario}
}
\newglossaryentry{iacconcepto}{
    type=tech,
    name={Infraestructura como Código},
    description={Práctica que permite definir infraestructura mediante archivos de configuración versionables}
}
\newglossaryentry{orquestacion}{
    type=tech,
    name={Orquestación},
    description={Sistema que automatiza la administración, escalamiento y operación de contenedores}
}
\newglossaryentry{microservicio}{
    type=tech,
    name={Microservicio},
    description={Componente autónomo que implementa una funcionalidad específica y puede escalarse, desplegarse y actualizarse de forma independiente}
}
\newglossaryentry{monolito}{
    type=tech,
    name={Monolito},
    description={Arquitectura en la que toda la funcionalidad del sistema está contenida en un único bloque de software}
}
\newglossaryentry{pipeline}{
    type=tech,
    name={Pipeline},
    description={Secuencia automatizada de pasos que ejecutan compilación, pruebas, análisis y despliegue de software}
}
\newglossaryentry{repositorio}{
    type=tech,
    name={Repositorio},
    description={Lugar donde se almacena y versiona el código fuente, documentación y archivos de un proyecto}
}
\newglossaryentry{seguridadinformatica}{
    type=tech,
    name={Seguridad informática},
    description={Conjunto de prácticas destinadas a proteger datos, sistemas y comunicaciones frente a accesos no autorizados}
}
\newglossaryentry{token}{
    type=tech,
    name={Token},
    description={Credencial digital utilizada para autenticar o autorizar a un usuario}
}
\newglossaryentry{workflow}{
    type=tech,
    name={Workflow},
    description={Secuencia de actividades que conforman un proceso funcional}
}
\newglossaryentry{devopsdef}{
    type=tech,
    name={DevOps},
    description={Conjunto de prácticas, principios y herramientas que integran el desarrollo de software y las operaciones de TI, con el objetivo de automatizar y optimizar los procesos de desarrollo, integración, despliegue y operación, promoviendo la colaboración continua y la entrega frecuente de software de calidad}
}
\newglossaryentry{hosting}{
    type=tech,
    name={Hosting},
    description={Servicio que proporciona infraestructura y recursos computacionales para alojar, ejecutar y mantener aplicaciones, sitios web o servicios digitales, permitiendo su acceso a través de Internet}
}
\newglossaryentry{pilatecnologica}{
    type=tech,
    name={Pila Tecnológica},
    description={Una pila tecnológica o \textit{tech stack} es el conjunto de herramientas de software utilizadas para desarrollar un proyecto.}
}


\newacronym{api}{API}{Application Programming Interface — Interfaz de Programación de Aplicaciones}
\newacronym{ci}{CI}{Continuous Integration — Integración Continua}
\newacronym{cd}{CD}{Continuous Deployment — Despliegue Continuo}
\newacronym{devops}{DevOps}{Development and Operations — Desarrollo y Operaciones}
\newacronym{ftp}{FTP}{File Transfer Protocol — Protocolo de Transferencia de Archivos}
\newacronym{gcp}{GCP}{Google Cloud Platform — Plataforma de Nube de Google}
\newacronym{iac}{IaC}{Infrastructure as Code — Infraestructura como Código}
\newacronym{json}{JSON}{JavaScript Object Notation — Notación de Objetos de JavaScript}
\newacronym{k8s}{K8s}{Kubernetes — Orquestador de Contenedores}
\newacronym{ui}{UI}{User Interface — Interfaz de Usuario}
\newacronym{ux}{UX}{User Experience — Experiencia de Usuario}


\glsaddall
