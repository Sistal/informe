\chapter*{Conclusiones}
\addcontentsline{toc}{chapter}{CONCLUSIONES}



% -----------------------------------------------------------

El desarrollo del proyecto permitió comprobar que la reingeniería de un sistema monolítico, carente de estructura y buenas prácticas, puede abordarse de manera efectiva mediante una arquitectura moderna basada en servicios desacoplados, contenedores y despliegue en la nube. El proyecto demuestra que se puede migrar desde una solución obsoleta hacia una plataforma más mantenible, escalable y alineada con estándares actuales de la industria del software.

La posibilidad de mejorar la calidad estructural del sistema mediante la adopción de una arquitectura cloud-native y una pila tecnológica moderna, fue validada a través de la implementación funcional del módulo de funcionarios. Dicha implementación evidenció mejoras concretas en la separación de responsabilidades, integridad de los datos y automatización del ciclo de vida del software, además de lograr implementar un módulo completo en un corto periodo de tiempo.

En cuanto a los alcances del trabajo, se logró diseñar una arquitectura completa y desplegable, acompañada de un modelo de datos normalizado, flujos de integración y despliegue continuo, y una organización del desarrollo basada en buenas prácticas de control de versiones y gestión de tareas.

Desde el punto de vista disciplinar, el proyecto aporta un caso práctico de reingeniería de software aplicada, integrando arquitectura de microservicios, DevOps y computación en la nube, lo que resulta relevante tanto para contextos académicos como profesionales. La experiencia obtenida refuerza la importancia de abordar los problemas de sistemas heredados desde una perspectiva arquitectónica y no únicamente tecnológica.

Dentro de las dificultades presentadas en el proyecto, se encuentran el proceso inicial de familiarización del sistema, la cual consumió la mayor parte del tiempo y requirió de muchas reuniones. También nos encontramos con diversos desafíos tecnológicos ya que muchas de las tecnologías aplicadas en este proyecto no se abordan dentro del transcurso de la carrera.

% Las conclusiones pueden incluir los resultados obtenidos en la investigación, comprobación o refutación de la hipótesis, recomendaciones que puedan ser útiles al problema de investigación, reflejando a su vez los alcances y limitaciones del trabajo, aportes al campo o disciplina del conocimiento y conclusiones generales.
% 
% Deben tener una redacción clara, concreta y directa; no son un resumen de la investigación.
